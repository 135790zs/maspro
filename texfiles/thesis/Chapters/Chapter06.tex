%************************************************
\chapter{Conclusion}\label{ch:conclusion}
%************************************************

In this report, the e-prop framework was applied on the TIMIT phone classification task, meeting the objectives listed in Chapter \ref{ch:introduction}.

First, the performance of the ALIF neuron was reproduced using the explicit e-prop equations.
Next, the STDP-LIF neuron was modified to the STDP-ALIF model that was experimentally verified to outperform ALIF.
Also, the Izhikevich neuron, which also shows STDP behavior, was shown to be unstable and performing worse than the ALIF and STDP-ALIF neurons.
This suggests that STDP does not provide an adequate neuron model by itself, but that e.g. spike frequency adaptation also needs to be taken into account.

Finally, the effect of stacking multiple layers was also examined, and did not improve the learning performance in this task.

E-prop remains a framework that offers high potential for biologically plausible learning algorithms for SNNs, which can be particularly well-suited for replicating intelligent behavior in low-power neuromorphic hardware.
