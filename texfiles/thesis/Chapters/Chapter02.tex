%************************************************
\chapter{Related work}\label{ch:relatedwork}
%************************************************

\section{Three-factor Hebbian learning}
    \begin{tcolorbox}[colback=orange]
    -3F Hebbian learning
      - STDP is observed
        - What (math, plot?), how biologically
      - But how can these synaptic changes lead to long-term behaviorial changes?
      - STDP requires modulatory signals (bailey2000heterosynaptic)

    \vspace{6cm}
    \end{tcolorbox}

    \subsection{Spike-timing dependent plasticity}
        \begin{tcolorbox}[colback=orange]
        - Clopath rule
        - R-STDP
        - (and other variants if they're relevant)

        \vspace{10cm}

        \end{tcolorbox}


    \subsection{Learning signal}
        \begin{tcolorbox}[colback=orange]
        - Biological plausilibity, (how does it happen in brain?)
        - Error-related negativity (see Bellec1)
        \vspace{10cm}

        \end{tcolorbox}

        \subsubsection{Broadcasting}

            \begin{tcolorbox}[colback=orange]
            - Broadcasting methods
              - Broadcast alignment (see Bellec1)
              - In brain
            \vspace{6cm}

            \end{tcolorbox}


    \subsection{Eligibility traces}
        \begin{tcolorbox}[colback=orange]
        - Why necessary?
        - Bioplausibility

        \vspace{12cm}

        \end{tcolorbox}

\section{Eligibility Propagation}

    \subsection{Network model}
        \begin{tcolorbox}[colback=orange]
        - Network model (formal, see Bellec)

        \vspace{14cm}

        \end{tcolorbox}

    \subsection{Neuron models}
        \begin{tcolorbox}[colback=orange]
        - LIF
        - ALIF
          - Mention GLIF (Bellec2)
        - (include figures)
        \vspace{18cm}

        \end{tcolorbox}

    \subsection{Learning procedure}

        \begin{tcolorbox}[colback=orange]
        - Neuron variables
        - Synapse variables

        \vspace{20cm}

        \end{tcolorbox}

    \subsection{Deriving e-prop from RNNs}

        \begin{tcolorbox}[colback=orange]
        - Explanation of BPTT.
        - Derivation from RNN and backprop.
        \vspace{20cm}

        \end{tcolorbox}

\section{Synaptic scaling}

    \begin{tcolorbox}[colback=orange]
    - Synaptic Scaling (in brain, if applicable)
    \vspace{10cm}

    \end{tcolorbox}

\section{Metaplasticity}

    \begin{tcolorbox}[colback=orange]
    - Metaplasticity (in brain, if applicable)
    \vspace{10cm}

    \end{tcolorbox}

\section{Network topology}

    \begin{tcolorbox}[colback=orange]
      - Network topology (e.g. multilayer) (but keep relevant)
      - Mainly focus on how brain does it. Cite often! Don't hypothesize on effects, just describe with sources.
      - Also denote a subsection on effects of topologies in related ANNs (preferably (R)SNNs).
    \vspace{15cm}

    \end{tcolorbox}
