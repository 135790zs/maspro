%************************************************
\chapter{Method}\label{ch:method}
%************************************************
\section{Data Preprocessing}

	\subsection{The TIMIT speech corpus}
		TIMIT is a speech corpus that contains phonemically transcribed speech~\citep{garofolo1993darpa}.
		It contains 6300 sentences, 10 spoken by each of the 630 speakers.
		The male and female speakers lived in 8 different geographical regions in the United States during their childhood years, see Table~\ref{tab:dialects}.

		\begin{table}[h]
		    \myfloatalign
		    \begin{tabularx}{\textwidth}{lrrr} \toprule
		        \tableheadline{Dialect region} & \tableheadline{\#Male}
		        & \tableheadline{\#Female} & \tableheadline{Total} \\ \midrule
		        % Phantoms take care of right-alignment (works iff monospaced digits)
		        1 (New England)   & 31 (63\%) & 18 (27\%) &  49  \phantom{0}(8\%)  \\
		        2 (Northern)      & 71 (70\%) & 31 (30\%) & 102 (16\%) \\
		        3 (North Midland) & 79 (67\%) & 23 (23\%) & 102 (16\%) \\
		        4 (South Midland) & 69 (69\%) & 31 (31\%) & 100 (16\%) \\
		        5 (Southern)      & 62 (63\%) & 36 (37\%) &  98 (16\%) \\
		        6 (New York City) & 30 (65\%) & 16 (35\%) &  46  \phantom{0}(7\%)  \\
		        7 (Western)       & 74 (74\%) & 26 (26\%) & 100 (16\%) \\
		        8                 & 22 (67\%) & 11 (33\%) &  33  \phantom{0}(5\%)  \\
		        \midrule
		        All  & 438 (70\%) & 192 (30\%) & 630 (100\%) \\
		        \bottomrule
		    \end{tabularx}
		    \caption[TIMIT Dialect Regions]{Distibution of speakers' dialect regions and sexes. Speakers of dialect region 8 moved around a lot during their childhood.}  \label{tab:dialects}
		\end{table}

		The sentence text can be categorized into 2 \emph{dialect} sentences, 450 \emph{phonetically compact} sentences, and 1890 \emph{phonetically-diverse} sentences.

		The dialect sentences, which are spoken by all speakers, are designed to expose the dialectical variants of the speakers.
		The phonetically compact sentences are designed to include many pairs of phones.
		The phonetically diverse sentences are taken from the Brown Corpus~\citep{kucera1967computational} and the Playwrights Dialog~\citep{hultzsch1964tables} in order to maximize the number of allophones (\ie, different phones used to pronounce the same phoneme).
		Table~\ref{tab:types} lists an overview of the distribution of the number of speakers per sentence type.

		\begin{table}[h]
		    \myfloatalign
		    \begin{tabularx}{\textwidth}{lrrrr} \toprule
		        \tableheadline{Sentence type} & \tableheadline{\#Sentences}
		        & \tableheadline{\#Speakers} & \tableheadline{Total} \\ \midrule
		        % Phantoms take care of right-alignment (works iff monospaced digits)
		        Dialect & 2    & 630 & 1260\\
		        Compact & 450  & 7   & 3150 \\
		        Diverse & 1890 & 1   & 1890 \\
		        \midrule
		        Total   & 2342 &     & 6300 \\
		        \bottomrule
		    \end{tabularx}
		    \caption[TIMIT Sentence Types]{Distribution of sentence types.}  \label{tab:types}
		\end{table}

		Each of the sentences is encoded in wavefile (\texttt{.wav}) format, and comes with \todo{synonym} a corresponding text file indicating what phones are pronounced in the wavefile, and between which sample points.

	\subsection{Engineering features}
