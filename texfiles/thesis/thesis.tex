\RequirePackage{silence} % :-\
    \WarningFilter{scrreprt}{Usage of package `titlesec'}
    \WarningFilter{scrreprt}{Activating an ugly workaround}
    \WarningFilter{titlesec}{Non standard sectioning command detected}
\documentclass[ twoside,openright,titlepage,numbers=noenddot,%1headlines,
                headinclude,footinclude,cleardoublepage=empty,abstract=on,
                BCOR=5mm,paper=a4,fontsize=11pt
                ]{scrreprt}

\input{classicthesis-config}

\addbibresource{Bibliography.bib}

\usepackage{todonotes} %
%\hyphenation{put special hyphenation here}

\begin{document}
\frenchspacing
\raggedbottom
\selectlanguage{american} % american ngerman
%\renewcommand*{\bibname}{new name}
\setbibpreamble{}
\pagenumbering{roman}
\pagestyle{plain}

%********************************************************************
% Frontmatter
%*******************************************************
% \include{FrontBackmatter/DirtyTitlepage}
%*******************************************************
% Titlepage
%*******************************************************
\begin{titlepage}
    %\pdfbookmark[1]{\myTitle}{titlepage}
    % if you want the titlepage to be centered, uncomment and fine-tune the line below (KOMA classes environment)
    \begin{addmargin}[-1cm]{-3cm}
    \begin{center}
        \includegraphics[width=10cm]{gfx/rugfsehor} \\ \medskip
        \large

        \hfill

        \vfill

        \begingroup
            \color{CTtitle}\spacedallcaps{\myTitle} \\ \bigskip
        \endgroup

        \spacedlowsmallcaps{\myName}

        \vfill

        % \mySubtitle \\ \medskip
        \myDegree \\
        % \myDepartment \\
        \myFaculty \\
        \myUni \\ \bigskip
        Supervised by Dr. Herbert Jaeger \& Dr. Marco Wiering \\ \bigskip

        \myTime

        \vfill

    \end{center}
    % \begin{flushright}
    % \end{flushright}
  \end{addmargin}
\end{titlepage}

\include{FrontBackmatter/Titleback}
% \cleardoublepage\include{FrontBackmatter/Dedication}

% \cleardoublepage\include{FrontBackmatter/Foreword}

% \cleardoublepage%*******************************************************
% Abstract
%*******************************************************
%\renewcommand{\abstractname}{Abstract}
\pdfbookmark[1]{Abstract}{Abstract}
% \addcontentsline{toc}{chapter}{\tocEntry{Abstract}}
\begingroup
\let\clearpage\relax
\let\cleardoublepage\relax
\let\cleardoublepage\relax

\chapter*{Abstract}
Spiking neural networks (SNNs) in neuromorphic systems are more energy efficient compared to deep learning--based methods, but there is no clear competitive learning algorithm for training such SNNs.
Eligibility propagation (e-prop) offers an efficient and biologically plausible way to train competitive recurrent SNNs in low-power neuromorphic hardware.
In this report, previous performance of e-prop on a speech classification task is reproduced, and the effects of inluding STDP-like behavior is analyzed.
A new neuron model that shows STDP-like behavior performs competitively, but this is not the case in the new Izhikevich e-prop neuron.
Finally, e-prop implemented in a single-layer recurrent SNN consistently outperforms a multi-layer variant.

\endgroup

\vfill

% \cleardoublepage\include{FrontBackmatter/Publications}
% \cleardoublepage\include{FrontBackmatter/Acknowledgments}
\cleardoublepage\include{FrontBackmatter/Contents}

%********************************************************************
% Mainmatter
%*******************************************************
\cleardoublepage
\pagestyle{scrheadings}
\pagenumbering{arabic}
\cleardoublepage


% %************************************************
\chapter{Introduction}\label{ch:introduction}
%************************************************

A primary goal of artificial intelligence is to develop systems that exhibit intelligent behavior.
During the 1980s, with the popularization of backpropagation \citep{rumelhart1986learning} and trainable Hopfield networks \citep{hopfield1982neural}, the focus of the field shifted from expert systems and symbolic reasoning to \emph{connectionist} approaches, such as artificial neural networks (ANNs).
ANNs are networks of small computational units that can be trained to perform specific pattern recognition tasks.
These networks are based loosely on the human brain.

As computing power and data storage capabilities increased exponentionally, and the rise of the internet provided abundant training data, ANNs have become a dominant field in artificial intelligence in the context of deep learning (DL).
This has particularly been the case during the the 2010s, when GPUs were increasingly used to train deep neural networks.
During the same period, convolutional neural networks (CNNs) and recurrent neural networks (RNNs) approached or exceeded human level performance in some areas \citep{schmidhuber2015deep}.
CNNs were also inspired by neuroscience; the connectivity pattern between units in a CNN resembles the organization of the primate visual cortex \citep{hubel1968receptive}.

However, DL-based methods are starting to show diminishing returns; training some state-of-the-art models can require so much data and computing power that only a small number of organizations has the resources to train and deploy them.
The computational processes of self-driving cars, for example, consume on the order of a thousand watts.
IFLYTEK-CV, which is one of the best-performing systems in the LFW challenge for facial recognition, was trained using a dataset of 3.8 million face images of 85 thousant individuals.\todo{from vigneron2020, but cite original}
ResNet \todo{cite} has been trained for 3 weeks on a 8-GPU server \todo{specify} consuming about 1 GWh. \todo{from vigneron2020, but cite original}
As a more extreme example, training the 11-billion parameter version of Google's T5 model \citep{raffel2019exploring} is estimated to cost more than \$1.3 million per run \citep{sharir2020cost}.
This contrasts strongly with the energy consumption of the human brain, which is made up of around 86 billion neurons \citep{azevedo2009equal} and 100--500 trillion synapses \citep{drachman2005we}, consumes approximately \SI{20}{\watt} \citep{sokoloff1960metabolism,drubach2000brain}, and does not require as much data to learn patterns.
This difference in power consumption is crucial for computations in mobile low-power or small-scale devices.

One reason why the human brain is more energy-efficient is that its computational function is realized in an analog and massively parallel physical substrate \citep{a2017parallel}, in which neurons communicate through sparsely occurring spikes \citep{bear2020neuroscience}.
Connections in DL models, on the other hand, are represented by multiplications between the often large matrices of the floating-point weight values of these connections and the activation values of the efferent units \citep{lecun2015deep}.
Backpropagation, which has become the de-facto standard for training DL models, is a biologically implausible learning algorithm that trains models by propagating the error back into the network, further raising computational costs.

Spiking neural networks (SNNs) are another step towards biological plausibility of connectionist models, and their concept dates back to the 1980s \citep{hopfield1982neural}.
SNNs use neurons that do not relay continuous activation values at every propagation cycle, but spike once when they reach a threshold value.
However, spike-based activation is not differentiable, gradient descent is not as effective as in ANNs to minimize the loss.
Consequentially, the lack of suitable training methods has limited the popularity of SNNs.

Neuromorphic computing is an emerging technology that, like the human brain, performs computation in an analog substrate.
This technology has the potential to offer more energy-efficient computation than the von Neumann architecture that is standard in training DL models.
Due to the centralized nature of von Neumann computers, emulated SNNs do not benefit from this energy efficiency as much as networks of biological neurons.
However, neuromorphic systems are decentralized, analog, and massively parallel, and may be orders of magnitude more efficient in running SNNs than digital computers.
This requires a learning algorithm that is both spatially and temporally local (\ie, neurons and synapses can only change their state based on information that is available at the same time step and immediately adjacent to that neuron or synapse).
However, finding such a learning algorithm that performs competitively with continuously-valued neural networks is an active field of research.

% It has long been understood that spike-timing-dependent plasticity (STDP), which is related to Hebbian learning plays a major role in biological learning and memory consolidation \todo{cite}.
A major factor in biological learning and memory consolidation is classical Hebbian learning, which is often summarized by ``cells that fire together, wire together'', but only if there is a causal relationship between these cells.
This causal relationship in biological neural networks is caused by a synapse that connects cells.
Applying Hebbian learning in a spiking neural network will generally lead to a positive feedback loop, because cells that wire together, fire together, and thus results in runaway excitation.

To learn any task or train any response, such as in classical conditioning, a learning signal, indicating a reward or punishment, should be present.
In the human brain, the neurotransmitter dopamine is behaviorally related to novelty and reward prediction \citep{li2003dopamine,schultz2007behavioral}.
Three-factor Hebbian learning incorporates such a learning signal, such that when the learning signal is positive, the connection between two interacting cells reinforces, and when it is negative, it weakens \citep{fremaux2016neuromodulated}.
Including such a learning signal has been demonstrated to improve the performance of Hebbian learning \citep{porr2007learning}.

Spike-timing-dependent plasticity (STDP) is a learning algorithm that is closely related to classical Hebbian learning.
In STDP, the temporal difference ... FINISH THIS

% Three-factor Hebbian learning has inspired many SNN learning algorithms



\begin{tcolorbox}[colback=orange]
- Brief paragraph on Hebbian learning

\end{tcolorbox}



Recent research by \cite{bellec2020solution} suggests that \emph{eligibility propagation} (e-prop), which is a spatially and temporally local learning algorithm for SNNs, may be a promising approach to train biologically plausible SNNs in a neuromorphic architecture.
\begin{tcolorbox}[colback=orange]

- E-prop approximates BPTT using RSNNs by using eligibility traces and learning signals. Also mathematical link (only intuition!)
- Multi-layer architectures have shown (?) to improve performance in ANNs, RNNs, and SNNs. Argue why, briefly. This is the motivation for examining multi-layer e-prop models.

\end{tcolorbox}

In this report, results from \cite{bellec2020solution} are reproduced, an experimental validation is performed on two new e-prop neuron models proposed by \cite{traub2020learning}, and the performance of e-prop in a multi-layer network is examined.


\begin{tcolorbox}[colback=orange]

- This paper examines the effects of enhancements that may improve the performance of e-prop. Some of these are used succesfully in DL. (Argue scientifically why these might improve performance)
    - Multilayer. Mention how MLPs were breakthrough on perceptrons.
    - Other neuron types
    - Regularization that's also observed in brain (e.g. synaptic scaling)


\end{tcolorbox}

%************************************************
\chapter{Method}\label{ch:method}
%************************************************
\section{Data Preprocessing}

	\subsection{The TIMIT speech corpus}
		TIMIT is a speech corpus that contains phonemically transcribed speech~\citep{garofolo1993darpa}.
		It contains 6300 sentences, 10 spoken by each of the 630 speakers.
		The male and female speakers lived in 8 different geographical regions in the United States during their childhood years, see Table~\ref{tab:dialects}.

		\begin{table}[h]
		    \myfloatalign
		    \begin{tabularx}{\textwidth}{lrrr} \toprule
		        \tableheadline{Dialect region} & \tableheadline{\#Male}
		        & \tableheadline{\#Female} & \tableheadline{Total} \\ \midrule
		        % Phantoms take care of right-alignment (works iff monospaced digits)
		        1 (New England)   & 31 (63\%) & 18 (27\%) &  49  \phantom{0}(8\%)  \\
		        2 (Northern)      & 71 (70\%) & 31 (30\%) & 102 (16\%) \\
		        3 (North Midland) & 79 (67\%) & 23 (23\%) & 102 (16\%) \\
		        4 (South Midland) & 69 (69\%) & 31 (31\%) & 100 (16\%) \\
		        5 (Southern)      & 62 (63\%) & 36 (37\%) &  98 (16\%) \\
		        6 (New York City) & 30 (65\%) & 16 (35\%) &  46  \phantom{0}(7\%)  \\
		        7 (Western)       & 74 (74\%) & 26 (26\%) & 100 (16\%) \\
		        8                 & 22 (67\%) & 11 (33\%) &  33  \phantom{0}(5\%)  \\
		        \midrule
		        All  & 438 (70\%) & 192 (30\%) & 630 (100\%) \\
		        \bottomrule
		    \end{tabularx}
		    \caption[TIMIT Dialect Regions]{Distibution of speakers' dialect regions and sexes. Speakers of dialect region 8 moved around a lot during their childhood.}  \label{tab:dialects}
		\end{table}

		The sentence text can be categorized into 2 \emph{dialect} sentences, 450 \emph{phonetically compact} sentences, and 1890 \emph{phonetically-diverse} sentences.

		The dialect sentences, which are spoken by all speakers, are designed to expose the dialectical variants of the speakers.
		The phonetically compact sentences are designed to include many pairs of phones.
		The phonetically diverse sentences are taken from the Brown Corpus~\citep{kucera1967computational} and the Playwrights Dialog~\citep{hultzsch1964tables} in order to maximize the number of allophones (\ie, different phones used to pronounce the same phoneme).
		Table~\ref{tab:types} lists an overview of the distribution of the number of speakers per sentence type.

		\begin{table}[h]
		    \myfloatalign
		    \begin{tabularx}{\textwidth}{lrrrr} \toprule
		        \tableheadline{Sentence type} & \tableheadline{\#Sentences}
		        & \tableheadline{\#Speakers} & \tableheadline{Total} \\ \midrule
		        % Phantoms take care of right-alignment (works iff monospaced digits)
		        Dialect & 2    & 630 & 1260\\
		        Compact & 450  & 7   & 3150 \\
		        Diverse & 1890 & 1   & 1890 \\
		        \midrule
		        Total   & 2342 &     & 6300 \\
		        \bottomrule
		    \end{tabularx}
		    \caption[TIMIT Sentence Types]{Distribution of sentence types.}  \label{tab:types}
		\end{table}

		Each of the sentences is encoded in wavefile (\texttt{.wav}) format, and comes with \todo{synonym} a corresponding text file indicating what phones are pronounced in the wavefile, and between which sample points.

	\subsection{Engineering features}

% %************************************************
\chapter{Method}\label{ch:method}
%************************************************
\section{Data Preprocessing}

	\subsection{The TIMIT speech corpus}
		TIMIT is a speech corpus that contains phonemically transcribed speech~\citep{garofolo1993darpa}, comprising 6300 sentences, 10 spoken by each of the 630 speakers.
		To include a broad range of dialects all speakers lived in 8 different geographical regions in the United States (as categorized in \cite{labov2008atlas}) during their childhood years.
		Table~\ref{tab:dialects} breaks down the precise composition of the dialect distribution.

		\begin{table}[ht]
		    \myfloatalign
		    \begin{tabularx}{\textwidth}{lrrr} \toprule
		        \tableheadline{Dialect region} & \tableheadline{\#Male}
		        & \tableheadline{\#Female} & \tableheadline{Total} \\ \midrule
		        % Phantoms take care of right-alignment (works iff monospaced digits)
		        1 (New England)   & 31 (63\%) & 18 (27\%) &  49  \phantom{0}(8\%)  \\
		        2 (Northern)      & 71 (70\%) & 31 (30\%) & 102 (16\%) \\
		        3 (North Midland) & 79 (67\%) & 23 (23\%) & 102 (16\%) \\
		        4 (South Midland) & 69 (69\%) & 31 (31\%) & 100 (16\%) \\
		        5 (Southern)      & 62 (63\%) & 36 (37\%) &  98 (16\%) \\
		        6 (New York City) & 30 (65\%) & 16 (35\%) &  46  \phantom{0}(7\%)  \\
		        7 (Western)       & 74 (74\%) & 26 (26\%) & 100 (16\%) \\
		        8                 & 22 (67\%) & 11 (33\%) &  33  \phantom{0}(5\%)  \\
		        \midrule
		        All  & 438 (70\%) & 192 (30\%) & 630 (100\%) \\
		        \bottomrule
		    \end{tabularx}
		    \caption[TIMIT Dialect Regions]{Distribution of speakers' dialect regions and sexes. Speakers of the innominate dialect region 8 relocated often during their childhood.}  \label{tab:dialects}
		\end{table}

		The sentence text can be categorized into 2 \emph{dialect} sentences, 450 \emph{phonetically compact} sentences, and 1890 \emph{phonetically diverse} sentences.

		The dialect sentences, which are spoken by all speakers, are designed to expose the dialectical variants of the speakers.
		The phonetically compact sentences are designed to include many pairs of phones.
		The phonetically diverse sentences are taken from the Brown Corpus~\citep{kucera1967computational} and the Playwrights Dialog~\citep{hultzsch1964tables} in order to maximize the number of allophones (\ie, different phones used to pronounce the same phoneme).
		Table~\ref{tab:types} lists an overview of the distribution of the number of speakers per sentence type.

		\begin{table}[ht]
		    \myfloatalign
		    \begin{tabularx}{\textwidth}{lrrrr} \toprule
		        \tableheadline{Sentence type} & \tableheadline{\#Sentences}
		        & \tableheadline{\#Speakers} & \tableheadline{Total} \\ \midrule
		        % Phantoms take care of right-alignment (works iff monospaced digits)
		        Dialect & 2    & 630 & 1260\\
		        Compact & 450  & 7   & 3150 \\
		        Diverse & 1890 & 1   & 1890 \\
		        \midrule
		        Total   & 2342 &     & 6300 \\
		        \bottomrule
		    \end{tabularx}
		    \caption[TIMIT Sentence Types]{Distribution of sentence types.}  \label{tab:types}
		\end{table}

		Each of the sentences is encoded in as a waveform signal in \texttt{.wav} format, and is accompanied by a corresponding text file indicating which phones are pronounced in the waveform, and between which pairs of sample points.

	\subsection{Data splitting}
		The TIMIT dataset is split into a training, validation and testing set as in \cite{graves2005framewise} and \cite{bellec2020solution}.
		The training set is used to train the network synaptic weights according to the e-prop algorithm.
		The validation set is used to obtain a well-performing set of hyperparameters, and to anneal the learning rate (see Section \ref{sec:lr_annealing}).
		The testing set is used to evaluate the performance of the network after the hyperparameters are obtained.

		The TIMIT corpus documentation offers a suggested partitioning of the training and testing data, which is based on the following criteria:
		\begin{enumerate}
			\item 70\%--80\% of the data is used for training, and the remaining 20\%--30\% for testing.
			\item No speaker appears in both the training and testing portions.
			\item Both subsets include at least 1 male and 1 female speaker from every dialect region.
			\item There is a minimal overlap of text material in the two subsets.
			\item The test set should contain all phonemes in as many allophonic contexts as possible.
		\end{enumerate}
		In accordance with these criteria, the TIMIT corpus includes a ``core'' test set that contains 2 male speakers and 1 female speaker from each dialect, summing up to 24 speakers.
		Each of these speakers read a different set of 5 phonetically compact sentences, and 3 phonetically diverse sentences that were unique for each speaker.
		Consequently, the test set comprises 192 sentences ($24\times(5+3)$) and was selected such that it contains at least one occurrence of each phoneme.
		In this report, the TIMIT core test set is used, thereby meeting the criteria listed above.

		The remaining 4096 sentences are randomly partitioned into 3696 training sentences and 400 validation sentences.

	\subsection{Engineering features}

		In this subsection, we describe the preprocessing pipeline as in \cite{fayek2016}, which can be summarized by applying a pre-emphasis filter on the waveforms, then slicing the waveform in short frames, taking their short-term power spectra, computing 26 filterbanks, and finally obtain 12 Mel-Frequency Cepstrum Coefficients (MFCCs).
		We align these MFCCs with the phones found in the TIMIT dataset.
		An example of a waveform signal is given in Figure \ref{fig:signal}.
			\begin{figure}[ht]
				\centering
			    \includegraphics[width=.45\linewidth]{gfx/signal}
			    \label{fig:signal}
			    \caption{A raw waveform signal from the TIMIT dataset.}
			\end{figure}

		\paragraph{Pre-emphasis}

			In speech signals, high frequencies generally have smaller magnitudes than lower frequencies.
			To balance the magnitudes over the range of frequencies in the signal, we apply a pre-emphasis filter $y(t)$ on the waveform signal $x(t)$ defined in Equation \ref{eq:pre_emphasis}.
			\begin{equation}\label{eq:pre_emphasis}
				y(t) = x(t) - 0.97x(t-1)
			\end{equation}
			This procedure yields the additional benefit of improving the signal-to-noise ratio.
			An example of a pre-emphasized signal is given in Figure \ref{fig:signalemph}.
			\begin{figure}[ht]
				\centering
			    \includegraphics[width=.45\linewidth]{gfx/signalemph}
			    \label{fig:signalemph}
			    \caption[Pre-emphasis]{A signal after the pre-emphasis filter of Equation \ref{eq:pre_emphasis} was applied to it.}
			\end{figure}

		\paragraph{Framing}
			The waveforms, which are sampled at a rate $f_s$ of \SI{16}{\kHz}, cannot be directly used as input to the model, because they are too long---a typical sentence waveform contains in the order of tens of thousands of samples.
			Furthermore, the samples are not very informative, because they represent the sound wave of the uttered sound.
			These sounds are filtered by the shape of the vocal tract, which manifests itself in the envelope of the short time power spectrum of the sound.
			This power spectrum representation describes the power of the frequency components of the signal over a brief interval.
			We assume the frequency components to be stationary over short intervals---in contrast to the full sentence, which carries its meaning because it is non-stationary.
			Therefore, we transform the waveform signals into series of frequency coefficients of short-term power spectra.
			To obtain multiple short-term power spectra over the duration of the waveform, we slice it up into brief overlapping frames.


			Every 160 samples (equivalent to \SI{10}{ms}) of a pre-emphasized signal we take an interval frame of 400 samples (equivalent to \SI{25}{ms}).
			This means that the frames overlap by \SI{25}{ms}.
			The waveform is zero-padded such that the last frame also has 400 samples.
			By this process, we obtain signal frames $x_i(n)$, where $n$ ranges over 1--400, and $i$ ranges over the number of frames in the waveform.

			Then, we apply a Hamming window with the form
			\begin{equation}
				w\left[n\right] = a_0 - a_1\cos\left(\frac{2\pi n}{N-1}\right),
			\end{equation}
			where $N$ is the window length of 400 samples, $0 \leq n < N$, $a_0 = 0.53836$, and $a_1 = 0.46164$.
			A plot of this window is given in Figure \ref{fig:hamming}.
			\begin{figure}[ht]
				\centering
			    \includegraphics[width=.45\linewidth]{gfx/hamming}
			    \label{fig:hamming}
			    \caption{The magnitudes of the DFT of a frame.}
			\end{figure}
			This window is applied to reduce the spectral leakage, which manifests itself though sidelobes in the power spectra.
			Applying the Hamming window reduces the sidelobes to near-equiripple conditions \citep{SASPWEB2011}.\todo{plot for illustration}.

		\paragraph{Short-term power spectra}

			We obtain the power spectra $P_i$ for each frame by first taking the absolute $K$-point discrete Fourier transform (DFT) of the frame samples $x_i(n)$\todo{don't bother with eqn, just call mathbb(F)}
			\begin{equation}\label{eq:magframes}
				X_k = \left|\sum_{n=0}^{N-1}x_i(n)\cdot e^{-\frac{i2\pi}{N}kn}\right|,
			\end{equation}
			where $K=512$.
			This yields the magnitudes of the DCT of the frames (an example is illustrated in Figure \ref{fig:magframes}).
			\begin{figure}[ht]
				\centering
			    \includegraphics[width=.45\linewidth]{gfx/magframes}
			    \label{fig:magframes}
			    \caption{The magnitudes of the DFT of a frame.}
			\end{figure}

			We obtain the power spectrum using the equation

			\begin{equation}\label{eq:powframes}
				P = \frac{{X_k}^2}{K},
			\end{equation}
			an example of which is shown in Figure \ref{fig:powframes}.

			\begin{figure}[ht]
				\centering
			    \includegraphics[width=.45\linewidth]{gfx/powframes}
			    \label{fig:powframes}
			    \caption{A power spectrum of a frame.}
			\end{figure}

		\paragraph{Mel filterbank}

			We then transform the short-term power spectra to Mel-spaced filterbanks.
			The Mel scale is a scale of pitches that are perceptually equal in distance \citep{stevens1937scale}.
			This is in contrast to the frequency measurement, in which the human cochlea can better distinguish lower frequencies better than higher ones.
			The aim of converting to the Mel scale is to make every filterbank coefficient feature equally informative, thereby improving the learning performance of the model.

			The Mel-spaced filterbank is a set of 40 triangular filters that we apply to each frame in $P$.

			To compute the Mel-spaced filterbank we choose lower and upper band edges of \SI{0}{\Hz} and $f_s/2 = \SI{8}{\kHz}$, respectively, and convert these to Mels using
			\begin{equation}
				m(f) = 2595\log_{10}\left(1 + \frac{f}{700}\right),
			\end{equation}
			where $f$ is the frequency in $\SI{}{\Hz}$.
			We obtain a lower band edge of 0 Mels and an upper band edge of approximately 2835 Mels.

			We begin obtaining the 40 filterbanks by spacing 42 points $\mathbf{m}$ linearly between these bounds (inclusive).
			Hence, we obtain 42 points spaced exclusively between the bounds.

			Then, we convert each point $m$ back to \SI{}{\Hz} using
			\begin{equation}
				f = 700\left(10^{m/2595}-1\right).
			\end{equation}
			We round each resulting Mel-spaced frequency $f$ to their nearest Fourier transform bin $b$ using
			\begin{equation}
				b = \lfloor(K+1)\mathbf{f}/fs\rfloor
			\end{equation}

			The resulting 40 filterbanks with their corresponding Mels and frequencies are listed in Table \ref{tab:mels}.

			The $i\textsuperscript{th}$ filter in filterbank $H_i$ is a triangular filter that has its lower boundary at $b_{i}$ \SI{}{\Hz}, its peak at $b_{i+1}$ \SI{}{\Hz}, and its upper boundary at $b_{i+2}$ \SI{}{\Hz}.\todo{not sure}
			For other frequencies, they are 0.
			Therefore, the filterbank can be described by
			\begin{equation}
				H_i(k) = \begin{cases}
					0 & k<b_i\\
					\frac{k-b_i}{b_{i+1}-b_i} & b_i\leq k < b_{i+1} \\
					1 & k = b_{i+1} \\
					\frac{b_{i+2} - k}{b_{i+2}-b_{i+1}} & b_{i+1} < k \leq b_{i+2}\\
					0 & b_{i+2} < k
				\end{cases},
			\end{equation}
			where $0 \leq k \leq \frac{K}{2}$.
			These Mel-spaced filters are shown in Figure \ref{fig:filterbank}.
			\begin{figure}[ht]
				\centering
			    \includegraphics[width=.45\linewidth]{gfx/fbanks}
			    \label{fig:filterbank}
			    \caption{The Mel-spaced filterbanks.}
			\end{figure}

			We obtain a spectrogram $S$ of the frame (see \eg Figure \ref{fig:spectrogram}) after applying the filterbank to the short-term power spectrum.

			\begin{figure}[ht]
				\centering
			    \includegraphics[width=.45\linewidth]{gfx/spectrogram}
			    \label{fig:spectrogram}
			    \caption{An example of a spectrogram.}
			\end{figure}

		\paragraph{Mel-frequency cepstral coefficients}

			We observe that the coefficients in the spectrograms are strongly correlated, which would negatively impact the learning performance of the model \todo{why?}.

			Therefore, we apply the DCT again to decorrelate the coefficients and obtain the power cepstrum $C$ of the speech frame:\todo{do we take absolute?}

			\begin{equation}\label{eq:magframes}
				C(n) = \left|\sum_{n=0}^{N-1}S(n)\cdot e^{-\frac{i2\pi}{N}kn}\right|.
			\end{equation}

			We discard the first coefficient in $C$, because it is the average power of the input signal and therefore carries little meaning.
			We also discard coefficients higher than 13, because they represent only fast changes in the spectrogram and increase the complexity of the input signal while adding increasingly less meaning to it. \todo{source?}
			An example of the remaining MFCC components is shown in Figure \ref{fig:mfccs}.

			\begin{figure}[ht]
				\centering
			    \includegraphics[width=.45\linewidth]{gfx/mfcc}
			    \label{fig:mfccs}
			    \caption{An example of Mel-frequency cepstral coefficients that are given as input to the system.}
			\end{figure}

			Then, we balance the final MFCCs by centering each frame around the value 0.
			Next, the trailing frames that are labeled as `silent' are trimmed from the end of the input and target sequences.
			Finally, to reflect\todo{better wording: re-approximate?} the speed of the original waveform signal, the input sequences are stretched by a factor of 5, interpolating linearly between frames.
			The target sequences are also stretched by this factor, but proximally interpolated to retain its one-hot encoding.
			An example of the final MFCCs is given in \ref{fig:source_mfcc_target}.

		\paragraph{Target output}

			The target output of the model is a frame-wise representation of the phones that are uttered in a sentence.
			The TIMIT corpus contains text files indicating in what order phones occur in a sentence, and their starting and ending sample points.

			These phones are discretized into frames such that they align correctly with the MFCCs.
			They are represented in one-hot vector encoding.
			Since the dataset contains 61 different phones, this is also the length of these vectors.

			Figure \ref{fig:source_mfcc_target} illustrates the waveform data and its frame-wise aligned MFCCs and target output.

			\todo{side-by-side with original text and phonemes, label as fig:source\_mfcc\_target}



		\begin{figure}[ht]
		    \centering
		    \includegraphics[width=.45\linewidth]{gfx/signal}\\
		    \includegraphics[width=.45\linewidth]{gfx/mfcc}\\
		    % \includegraphics[width=.45\linewidth]{gfx/target}
		    \label{fig:source_mfcc_target}
		    \caption{An alignment of a sample signal with its MFCCs and target phones.}
		\end{figure}

\section{Enhancing e-prop}

	\begin{tcolorbox}[colback=orange]
	preamble

	\end{tcolorbox}

	\subsection{Multi-layer architecture}
		\begin{tcolorbox}[colback=orange]

		- English, visual and formal descriptions.

		\end{tcolorbox}

		The multi-layer e-prop architecture can be described in the same formal model as its single-layer counterpart, in which the hidden state is based on temporally (i.e., online) and spatially locally available information at a neuron $j$:

        \begin{equation}
        \mathbf{h}^t_j = M\left(\mathbf{h}_j^{t-1}, \mathbf{z}^{t-1}, \mathbf{x}^t, \mathbf{W}_j\right).\tag{\ref{eq:model} revisited}
        \end{equation}
        For the multi-layer architecture, however, neurons in deeper layers no longer depend on the input, but on the observable states of the previous layer at the same time step, such that at every time step, a full pass through the network is made.
        We modify the indexing notation accordingly, in order to directly refer to neurons and weights in a particular layer $r \in [1\mathrel{{.}\,{.}}\nobreak R]$:
        \begin{equation}\label{eq:ml_model}
        \mathbf{h}^t_{rj} = \begin{cases}
        M\left(\mathbf{h}_{rj}^{t-1}, \mathbf{z}_r^{t-1}, \mathbf{x}^t, \mathbf{W}_{rj}\right)       & \mbox{if } r = 1\\
        M\left(\mathbf{h}_{rj}^{t-1}, \mathbf{z}_r^{t-1}, \mathbf{z}_{r-1}^t, \mathbf{W}_{rj}\right) & \mbox{otherwise,}
        \end{cases}
        \end{equation}
        where $\mathbf{h}^t_{rj}$ (resp. $z^t_{rj}$) is the hidden state (resp. observable state) of a neuron $j$ in layer $r$. and $\mathbf{W}_{rj} = \mathbf{W}^\text{in}_{rj} \cup \mathbf{W}^\text{rec}_{rj}$ is the set of afferent weights to neuron $j$ in layer $r$.

        Similarly, the observable state can be modeled by
        \begin{equation}\label{eq:ml_model_obs}
        z^t_{rj} = f\left(\mathbf{h}_{rj}^t\right)
        \end{equation}
        and the network output by
        \begin{equation}\label{eq:ml_model_obs}
        y^t_k = \kappa y^{t-1}_k + \sum_{j,r}W^\text{out}_{rkj}z_{rj}^t + b_k,
        \end{equation}
        where $W^\text{out}_{rkj}$ is a weight between neuron $j$ in layer $r$ and output neuron $k$.
        Note that the summation over $r$ entails that the output layer is connected to all neurons in all layers in the network.
        This allows trainable broadcast weights in earlier layers, such as those found in symmetric and adaptive e-prop.

        \paragraph{Multi-layer ALIF neurons}
        An ALIF neuron in a multi-layer architecture is similar to one in a single-layer architecture (see Section \ref{sec:alif}).
        The only difference, apart from the layer indexing, is its activity update.
        For a multi-layer ALIF neuron, the activity value is given by
        \begin{equation}\label{eq:ml_alifV}
        v^{t+1}_{rj} = \alpha v_{rj}^t + \sum_{i\neq j}W^\text{rec}_{rji}z_i^t + \sum_i W^\text{in}_{rji}I - z_{rj}^tv_
        \text{th},
        \end{equation}
        where
        \begin{equation}
        I = \begin{cases}
        	x^{t+1}_i       &\mbox{if } r = 1 \\
            z^{t+1}_{r-1,i} &\mbox{otherwise.}
            \end{cases}
        \end{equation}


	\subsection{Other neuron types}
		\begin{tcolorbox}[colback=orange]

		- Shoutout to Traub
		- Argue in favor of different models (refer to brain, simplicity vs. plausibility trade-off). E.g.: built-in refractory

		\end{tcolorbox}
		\subsubsection{STDP-LIF}
			\begin{tcolorbox}[colback=orange]

			- STDP-LIF (intuition, maths, graphs)
			- Also mention and motivate v-fix and psi-fix.
			- Mention Bellec's reset too.

			\end{tcolorbox}
		\subsubsection{Izhikevich neuron}
			\begin{tcolorbox}[colback=orange]

			- STDP-LIF (intuition, maths, graphs)

			\end{tcolorbox}

\section{Regularization}

	\begin{tcolorbox}[colback=orange]
	preamble: explain why this is used (bioplausibility (natural constraints) and generalizability)

	\end{tcolorbox}

	\subsection{Firing rate regularization}
		\begin{tcolorbox}[colback=orange]
		- What, why?
		- Bioplausible?

		\end{tcolorbox}
		Firing rate is implemented by adding a regularization term $E_\text{reg}$ to the loss function that penalizes neurons that have a firing rate that is too low or too high:
		\begin{equation}
			E_\text{reg} = \frac{1}{2}\sum_j\left(f^\text{target} - f^{\text{av}, t}_{rj}\right)^2,
		\end{equation}
		where $f^\text{target}$ is a target firing rate of \SI{10}{\Hz}, and
		\begin{equation}
		f^{\text{av},t}_{rj} = \frac{1}{t} z^{\text{total},t}_{rj}
		\end{equation}
		is the running average spike frequency, where $z^{\text{total},t}$ accumulates spikes emitted by neuron $j$ in layer $r$ up to (and including) time step $t$.
		Note that $z^{\text{total},0} = 0$, \ie, the accumulation resets at every new training sample.
		By implementing this sum as a hidden variable, e-prop remains an online training algorithm when firing rate regularization is implemented.

		To insert the regularization term into the e-prop framework, we compute the weight update that regularizes the firing rate toward $f^\text{target}$ through gradient descent, similarly to the main e-prop weight update (Equation \ref{eq:eprop_grd}):
		\begin{equation}
		\frac{\partial E_\text{reg}}{\partial z_{rj}^t} = \left(f^\text{target} - f^{\text{av}, t}_{rj}\right).
		\end{equation}
		Note that this regularization loss differs from the firing rate regularization in \cite{bellec2020solution}, in which the firing rate is calculated in an offline fashion, by retroactively computing the average firing rate based on all spikes instead of only accumulated spikes.
		Note also that in \cite{bellec2020solution}, $\frac{\partial E_\text{reg}}{\partial z_{rj}^t}$ is multiplied with the eligibility trace $e^t_{rji}$, as in Equation \ref{eq:eprop_grd} to obtain the weight update, whereas in this report, the eligibility trace is omitted, resulting in a number of benefits:
		\begin{enumerate}
			\item It allows silent neurons that have infrequently spiking afferent neurons to more easily increase their firing rate, because their low afferent eligibility traces no longer nullifies the regularization gradient, and thereby results in a better empirical learning performance;
			\item It is more efficient in emulations on von Neumann machines, because the element-wise multiplication of $\frac{\partial E_\text{reg}}{\partial z_{rj}^t}$ and the eligibility trace is a relatively large computation on the order $\Theta\!\left(n^2\right)$ that no longer needs to be computed.
			\item It is more intuitive, as only the gradient of the firing rate is used to compute the weight update.
		\end{enumerate}
		We apply the weight update $\Delta W_{rji}$ of the regularization gradient using
		\begin{equation}
		\Delta_\text{reg} W_{rji} = -\eta\ c_\text{reg}\sum_t\left(f^\text{target} - f^{\text{av}, t}_{rj}\right).
		\end{equation}
		Note that the regularization gradients can be combined and accumulated over time on the same synaptic variable as the normal gradients, facilitating practical implementation of the learning procedure in both software emulations and neuromorphic embeddings:
		\begin{equation}
		\Delta W_{rji} = -\eta\ \sum_t\left(c_\text{reg}\left(f^\text{target} - f^{\text{av}, t}_{rj}\right) + L^t_{rj}\cdot\bar{e}^t_{rji}).
		\end{equation}


	\subsection{Ridge regression}
		\begin{tcolorbox}[colback=orange]
		- What, why, how?
		- Bioplausible?

		\end{tcolorbox}

	\subsection{Weight decay}
		\begin{tcolorbox}[colback=orange]
		- What, why, how?
		- Bioplausible?

		\end{tcolorbox}


\section{Bidirectional network}
	\begin{tcolorbox}[colback=orange]
	- I/A
	- What, why, how?

	\end{tcolorbox}

\section{Optimizer}\label{sec:adam}
	\begin{tcolorbox}[colback=orange]
	- Show how Adam works, and how it replaces SGD

	\end{tcolorbox}


\section{Hyperparameter optimization}
	\begin{tcolorbox}[colback=orange]
	- What, why, how?

	\end{tcolorbox}


% %************************************************
\chapter{Results}\label{ch:results}
%************************************************
In this chapter, the learning performance and regularization behavior of the ALIF, STDP-ALIF, and Izhikevich neurons are compared.
Then, the effect of stacking multiple recurrent layers on the learning performance and speed is examined.
Figure \ref{fig:inoutpair} shows a typical classification result of a full validation sentence.

	\begin{figure}[ht]
	    \myfloatalign
	    \includegraphics[width=\linewidth]{gfx/InOutPair}
	    \caption[Input-output-target example.]{An example validation result using a trained ALIF model. The top row shows the standardized MFCC frames of a sentence changing over time. The second row shows the probability distributions of the frame-wise outputs of the model. The third row is the most likely phone. The last layer shows the target phones.}
	    \label{fig:inoutpair}
	  \end{figure}

\section{Comparing neur{}on models}
	\paragraph{Accuracy}
		The main outcome of the neuron model comparison is that in these results, the Izhikevich neuron type performs poorly compared to the ALIF and STDP-ALIF neuron models, which reach a similar classification performance.
		In Figure \ref{fig:percwrong} the Izhikevich neuron reaches a misclassification rate of 94.2\% on the test set, which is only slightly better than constantly guessing the most frequent class.
		The ALIF neuron reaches a test misclassification rate of 50\% in relatively few iterations, and the STDP-ALIF stably reaches 50\%, and likely lower if training was continued for longer.
		Note that the test performance was obtained on the model with the best validation accuracy (the used hyperparameters are listed in Table \ref{tab:hparams}).

		Figure \ref{fig:crossentropy} illustrates the decrease of the cross-entropy score, which for the ALIF and STDP-ALIF neurons is comparable to that of the misclassification rate.
		The cross-entropy of the Izhikevich neuron continues to decrease while its classification accuracy stalls, suggesting that it trains its bias toward more frequent phone classes rather than learning a relationship between input MFCCs and classes.

		\begin{figure}[bth]
		    \myfloatalign
		    \subfloat[Percentage of samples wrongly classified.]
		    {\label{fig:percwrong}\includegraphics[height=5cm, keepaspectratio]{gfx/percwrong}} \quad
		    \subfloat[Cross-entropy loss (log-scaled).]
		    {\label{fig:crossentropy}%
		        \includegraphics[height=5cm, keepaspectratio]{gfx/crossentropy}}
		    \caption[Classification performance for each of the three neuron models in a single-layer e-prop model.]{Classification performance on the validation data for each of the three neuron models in a single-layer e-prop model. The opaque lines indicate the running average of the real validation scores indicated by the transparent lines. The star symbols indicate the performances on the test set, with a misclassification rate of 92.4\% for the Izhikevich neuron, 50.3\% for the ALIF neuron, and 50\% for the STDP-ALIF neuron type.}\label{fig:sl-acc}
		\end{figure}

% testscores = {
% 	'ALIF': {1: (50.3, 1.772), 2: (64.5, 2.558), 3: (74.1, 3.345)},
% 	'STDP-ALIF': {1: (50.0, 1.751), 2: (65.3, 2.643), 3: (88.3, 4.779)},
% 	'Izhikevich': {1: (94.2, 6.794), 2: (88.2, 4.259), 3: (88.5, 4.161)}
% }

	\paragraph{Firing rate}
		Figure \ref{fig:freqs} illustrates the effect of the firing regularization term.
		It can be observed that the ALIF and STDP-ALIF neuron models are able to quickly change their mean spiking frequencies to the desired target frequency of \SI{10}{\Hz}, but the Izhikevich neuron increases past this rate to a spiking frequency of approximately \SI{18}{\Hz}.

		Figure \ref{fig:regerr} illustrates the decrease of the regularization error.
		The ALIF neuron appears to slightly better adapt to the regularization term.
		The Izhikevich neuron, again, underperforms significantly.
		\begin{figure}[bth]
		    \myfloatalign
		    \subfloat[Mean spiking frequency.]
		    {\label{fig:freqs}\includegraphics[height=5cm, keepaspectratio]{gfx/hz}} \quad
		    \subfloat[Regularization error.]
		    {\label{fig:regerr}%
		        \includegraphics[height=5cm, keepaspectratio]{gfx/regerr}}
		    \caption[Effect of firing rate regularization for each of the three neuron models.]{Effect of firing rate regularization on the validation data for each of the three neuron models.}\label{fig:sl-reg}
		\end{figure}

\section{Comparing network depth}
The comparison between the network depth in Figures \ref{fig:ml-pwrong-alif}--\ref{fig:ml-pwrong-izh} suggests that single-layer e-prop networks train considerably more efficiently and accurately than multi-layer e-prop networks.
This holds for all tested neuron types.
The cross-entropy error, spiking frequency, and regularization error are also better for single-layer networks (see Figure \ref{fig:ml-otherresults}).

\begin{figure}[bth]
    \myfloatalign
    \subfloat[ALIF model.]
    {\label{fig:ml-pwrong-alif}\includegraphics[height=5cm, keepaspectratio]{gfx/ml-percwrong-ALIF}} \quad
    \subfloat[STDP-ALIF model.]
    {\label{fig:ml-pwrong-stdpalif}%
        \includegraphics[height=5cm, keepaspectratio]{gfx/ml-percwrong-STDP-ALIF}} \\
    \subfloat[Izhikevich model.]
    {\label{fig:ml-pwrong-izh}\includegraphics[height=5cm, keepaspectratio]{gfx/ml-percwrong-Izhikevich}}
    \caption[Single- and multi-layer accuracy comparison]{Accuracy comparison on the validation data between single- and multi-layer e-prop models.}\label{fig:ml-percwrong}
\end{figure}

% %************************************************
\chapter{Discussion}\label{ch:discussion}
%************************************************

\paragraph{Interpretation of results}



\paragraph{Possible improvements}
    There are many possible ways of improving the performance or biological plausibility of e-prop that have not yet been considered in this report.
    For instance, a likely reason that the learning speed of the multi-layer architectures was slower than their single-layer counterparts is that the weights are poorly initialized.
    Empirical observation of the learning process suggested that spiking activity faded in deeper layers, because the spiking activity from a preceding layer is generally weaker than the input values the first layer receives.
    Higher weights in-between layers mitigate this fading activity, but require some search to find a good value.
    In this report, the firing rate regularization term approximated this value, but learning is more efficient with a better initialization, since intitial synaptic weights significantly affect the performance of STDP-based SNNs (Kim2020).

    Also in this report, certain parameters such as firing rate targets ($f^\text{target}$), activity leak ($\alpha$), threshold adaptivity ($\beta$), and feedback signals were constant for all neurons, except the threshold adaptivity, which could also be 0 to emulate non-adaptive LIF neurons.
    However, future research could examine the effects of sampling some of these parameters from a distribution for each neuron, thereby creating a more diverse population of neurons with different time scales.
    According to (Bellec2019), e-prop suggests that the experimentally found diverse time constants of the firing activity of populations of neurons in different brain areas (Bellec2019/30) are correlated with their capability to handle corresponding ranges of delays in temporal credit assignment for learning.
    Setting different values for these parameters per layer might also have a beneficial effect; \citet{ahmed1998estimates} suggested that deeper layers display slower and weaker adaptation rates than shallow layers.

    Random dropout of 80\% of the recurrent network weights led to better training and validation performances.
    This suggests that the effects of the topology within a layer might positively affect the learning process.
    In the brain, neurons tend to connect to nearby neurons.
    A simple lattice topology might better approximate the connectivity of the brain, decrease the computational complexity in both emulations in von Neumann machines, and allow easier on-chip implementations in neuromorphic hardware.
    Hierarchical clustering of neurons might also have a beneficial effect, as this has been demonstrated to improve R-STDP in SNNs (weidel2020) and address the scalability issue of SNNs (Zhang2020/6).
    Neuromorphic computation allows complex network operations (Yu2020/5), large-scale conductance-based SNNs (Zhang2020/8,9) and asynchronous communication in VLSIs through address event representation (Zenke2021/31,32).

    The connection topology might also change over time through a dynamic pruning and growing of weights and neurons.
    Here, the biological motivation is that the human brain prunes synaptic connections during early development (Elbez2020/8).
    (Elbez2020) demonstrated that 75\% of a SNN can be compressed while preserving its performance, but it is not clear if this can be applied in a biologically plausible way in the e-prop framework.
    However, integrating stochastic synaptic rewiring (Bellec2019/24) into an ALIF network can improve its short-term memory (Bellec2019).

    Finally, synaptic delay might improve the temporal processing power of an e-prop model.
    In this report, communication between neurons was transmitted as a spike over a synapse with a delay of 1 ms.
    This delay could be variable (\eg, beteen 1 and 10 ms), such that potentially informative past inputs are more accurately preserved in synaptic delays, rather than only in eligibility traces and activity loops.
    This resembles the variable physical length of myelinated biological synapses and number of nodes of Ranvier along them, affecting the conductance of the action potential from one neuron to another (Bean, The action potential in mammalian central neurons, 2007).

\paragraph{Future directions}
    As neuromorphic computing matures, neuroscience improves, and DL increasingly hits fundamental limitations, there is an exciting future for biologically plausible SNNs.
    There is much to gain from cross-fertilization between these fields.
    The popularity of DL was accelerated by accessible platforms to implement and deploy ANNs.
    Similar high-level simulation platforms are now in active development, which can integrate the typical behavior of memristive device models into crossbar architectures within DL systems (Lammie2020).

    Recent advances in neuromorphic computing indicate this increasing popularity.
    Neuromorphic architectures have been used for mapless navigation with 75 times lower power and better performance (Tang2020); as low-power solutions for localization and mapping of mobile robots (Tang2020/9), for planning (Tang2020/10), and control (Tang2020/11); and self-repairing SNN for fault detection (Zhang2020/7).
    While cross-fertilization between neuromorphics and quantum computing is starting to take place (Markovic2020/68,69), as quantum superposition and entanglement may be used to process information in parallel and in a high-dimensional state space (Markovic2020/101-103), more physics and material science is requires to build efficient neuromorphic architectures (Markovic2020).
    The same holds for the cross-fertilization between neuroscience and learning rules of biologically plausible SNNs.
    Nanodevices that emulate biological synapses with learning functions can benefit neuromorphics (Wang2020b/7-13), particularly the two-terminal memristor (Wang2020b/14-19). However, the learning principles of biological NNs are not explored enough to design engineering solutions (gorban2019unreasonable, Zhang2020/3).
    Feedback connections, for which the brain uses neurotransmitters, may become particularly problematic in large-scale neuromorphic systems.
    Another issue in analog computation is how to match the system's internal temporal processing to that of its inputs.
    Emulating neural dynamics on a physical substrate is more efficient but requires constraints to match the brain's timescales (Zenke2021/40).
    Future work on e-prop could explore a combination with attention-based models in order to cover multiple timescales (Bellec2019).




\begin{tcolorbox}[colback=orange]
Future research:
- Beta, rho, alpha, synaptic delay not constant but spread
- ahmed1998estimates finds that deeper layers have slower and weaker adaptation rates than shallow layers.
\end{tcolorbox}



% ********************************************************************
% Backmatter
%*******************************************************
\appendix
%\renewcommand{\thechapter}{\alph{chapter}}
\cleardoublepage
%************************************************
\chapter{Appendix}\label{ch:appendix}
%************************************************
\begin{table}[ht]
		    \myfloatalign
		    \begin{tabularx}{\textwidth}{rrr} \toprule
		        \tableheadline{Mels} & \tableheadline{Hz}
		        & \tableheadline{Filterbank} \\ \midrule
		        % Phantoms take care of right-alignment (works iff monospaced digits)
		        0    & 0\phantom{.0} & 0 \\
		        105  & 68.5   & 2 \\
 		        210  & 143.7  & 4 \\
 		        315  & 226.2  & 7 \\
		        420  & 316.8  & 10 \\
		        525  & 416.3  & 13 \\
		        630  & 525.5  & 16 \\
		        735  & 645.4  & 20 \\
		        840  & 777\phantom{.0} & 24 \\
		        945  & 921.5  & 29 \\
		        1050 & 1080.1 & 34 \\
		        1155 & 1254.4 & 40 \\
		        1260 & 1445.4 & 46 \\
		        1365 & 1655.3 & 53 \\
		        1470 & 1885.7 & 60 \\
		        1575 & 2138.6 & 68 \\
		        1680 & 2416.3 & 77 \\
		        1785 & 2721.2 & 87 \\
		        1890 & 3055.9 & 97 \\
		        1995 & 3423.3 & 109 \\
		        2100 & 3826.7 & 122 \\
		        2205 & 4269.5 & 136 \\
		        2310 & 4755.7 & 152 \\
		        2415 & 5289.4 & 169 \\
		        2520 & 5875.3 & 188 \\
		        2625 & 6518.6 & 209 \\
		        2730 & 7224.8 & 231 \\
		        2835 & 8000\phantom{.0} & 256 \\
				\bottomrule
		    \end{tabularx}
		    \caption[Filterbanks]{Conversion table between linearly spaced Mels and their corresponding frequencies and filterbank boundaries.}  \label{tab:mels}
		\end{table}


%********************************************************************
% Other Stuff in the Back
%*******************************************************
\cleardoublepage\include{FrontBackmatter/Bibliography}
% \cleardoublepage\include{FrontBackmatter/Declaration}
% \cleardoublepage\include{FrontBackmatter/Colophon}
% ************************************************************
\end{document}
% ********************************************************************

STRUCTURE
* Introduction
  - Artificial spiking neurons inspired by brain neurons (as in Bellec intro p1). Various theoretical advantages over ANNs.
  - Unclear how RSNNs can learn. In hope of crudely emulating biological neural circuits, try online, local and bioplausible method.
  - E-prop uses bio-inspired (which) methods that allow for online, local, bioplausible training and have shown to yield good results.
  - However, real neurons are stacked in layers. This paper generalizes e-prop to multiple layers. (short parallel to perceptrons vs, MLPs? will need some theoretical analysis like MLPs could nonlinearly separate. Time dynamics maybe?)
  - Increasing relevance for neuromorphic computing
  - ... synaptic scaling, Izhikevich etc will depend on whether it is implemented.

* METHOD
  - General top-down formulas (derivatives etc). Derive from RNNs. Explain Traub fix or Bellec reset if applicable.
  - Actual formulas of implementation (abstract away from bidir & epoch dimension) like Bellec, for LIF, ALIF, Izhikevich. Include derivations as proof.
  - TIMIT dataset explanation, including preprocessing
  - Regularization (FR, L2, synscaling, metaplasticity, etc)
  - Hyperparameter sweep (keep it objective and explain choices.)

* RESULTS
  - Results of TIMIT. Optionally: initialization & hparam robustness, tradeoffs, comparisons, generalizability, system behavior (e.g. spike freqs, time dynamics, per-phoneme performance, effect of MFCC derivatives), running cost in simulation and in neuromorphics (where only spikes cost energy).

* DISCUSSION
  -

* CONCLUSION
  - Short summary, key sentences from all previous. Logical connection to introduction.
  - Link to R-STDP with Traub fix.
  - Neuromorphic computing
  - Maybe: attempt to explain network interpretability. Subgraphs representing features.

* APPENDIX
  - Explain implementation, sweeping, and all details such that precise replication can be made.
  - Pseudocode


=================

METHOD
- Data Preprocessing
  - TIMIT overview
  - Audio to features
- Eligibility Propagation (basics, as in Bellec)
  - Network model
  - E-prop algorithm (LIF or ALIF depends on what's used in final model)
  - Derivation from RNN and backprop.
  - Regularization
    - FR (if applicable)
    - L2 (if applicable)
  - Adam (if applicable)
  - Bidirectional network (if applicable)
  - Traub or Bellec reset (if applicable)
- Extending the E-prop framework
  - Generalizing to multiple layers
  - Using Izhikevich neurons
- Hyperparameter optimization procedure
