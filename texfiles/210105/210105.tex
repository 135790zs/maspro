\documentclass[t]{beamer}

\setlength {\marginparwidth }{2cm}
\setlength {\parskip}{0cm}
\usepackage{todonotes}
\usepackage{siunitx}
\usepackage{subcaption}
\usepackage{apacite} 	% Use APA Citation
\presetkeys{todonotes}{inline}{}
\beamertemplatenavigationsymbolsempty
\usefonttheme[onlymath]{serif}

\usepackage{algpseudocode}
\usepackage{enumitem,amssymb, yfonts, bm}
\newlist{todolist}{itemize}{2}
\setlist[todolist]{label=$\square$}
\usepackage{pifont}
\newcommand{\cmark}{\ding{51}}%
\newcommand{\xmark}{\ding{55}}%
\newcommand{\done}{\rlap{$\square$}{\raisebox{2pt}{\large\hspace{1pt}\cmark}}%
\hspace{-2.5pt}}

% \usetheme{AnnArbor}
% \usetheme{Antibes}
% \usetheme{Bergen}
% \usetheme{Berkeley}https://www.sharelatex.com/project/5b12e1a4f84b363f6f336dab
% \usetheme{Berlin}
% \usetheme{Boadilla}
% \usetheme{boxes}
\usetheme{CambridgeUS}
% \usetheme{Copenhagen}
%\usetheme{Darmstadt}
% \usetheme{default}
%\usetheme{Frankfurt}
%\usetheme{Goettingen}
% \usetheme{Hannover}
% \usetheme{Ilmenau}
% \usetheme{JuanLesPins}
\setlength{\parskip}{10pt}

% \newcommand*\vc[1]%
% {\begin{pmatrix}#1\end{pmatrix}}

\newcommand*\vc[1]%
{\left(\begin{array}{cccc}#1\end{array}\right)}


\newcommand\eqdef{\ \mathrel{\overset{\makebox[0pt]{\mbox{\normalfont\scriptsize\rmfamily def}}}{=}}\ }

\title[Eligibility propagation]{Improving eligibility propagation using Izhikevich neurons in a multilayer RSNN.\\\vspace{10pt}
\large{Presentation 6: Reproducing Bellec's results}}

\author[Werner]{Werner~van~der~Veen\\\footnotesize\texttt({w.k.van.der.veen.2@student.rug.nl})}\date{\today}

\begin{document}

\begin{frame}
    \titlepage
\end{frame}

%======================================

%\begin{frame}
%    \tableofcontents
%\end{frame}

\section{Work done since previous meeting}
\begin{frame}{Work done since previous meeting}
  \begin{todolist}

    \item[\done] Implement Bellec's model;
    \item[\done] Number of major and minor fixes;
    \item[\done] Normalized input data per channel;
    \item[\done] Started writing and drafting thesis.
  \end{todolist}

\end{frame}

\begin{frame}{Performance}
  \begin{itemize}[label=--]

    \item Performance goal for e-prop with Bellec's configuration: 36.9\% misclassification rate on test set with fixed/random broadcast alignment;
    \item Currently: approaching that performance as I try and find differences in the code;
    \item Bellec doesn't report some of the hyperparameter settings, and uses inconsistent notation.
    \item I wrote additional visualization modules to pinpoint the errors.
    \item The system can now train and generalize well; just not down to the desired accuracy. I think this is simply a matter of tuning the learning rate.
  \end{itemize}

\end{frame}


\begin{frame}{Performance}
\center
  (show performance)
\end{frame}

\section{Planning}
\begin{frame}{Next steps}
  \begin{todolist}

    \item Reproduce Bellec's 36.9\% error rate. (est. 1--7 working days)
    \item Obtain results comparing the 1, 2, 3 layer architectures. (2--3 days).
    \item Obtain results comparing the (A)LIF, ALIF, STDP-LIF, and Izhikevich neurons for 1 layer. (3--4 days).
    \item Implement non-uniform synaptic delays (1 day).
    \item Implement metaplasticity (1 day).
    \item Obtain performance for non-uniform synaptic delay (2 days).
    \item ...and for metaplasticity (2 days).
    \item Concurrent with the above steps, draft and write parts of the method and introduction of the thesis.
  \end{todolist}

\end{frame}
\begin{frame}{Questions}
  \begin{itemize}[label=--]

    \item Which results are interesting? My proposal:
    \begin{tabular}{|l|l|l|}
    \hline
	Number of layers & 1, 2, 3.  (readout from all or last? (Bellec = all))\\
	\hline
	Neuron type & (A)LIF, ALIF, STDP-LIF, Izhikevich\\
	\hline
	Synaptic delay & 1, 1--8\\
	\hline
	\end{tabular}
	\item Combining the above?
	\item Draft the thesis top-down. Start with method?
	\item To what extent focus on peripheral topics (e.g. Hebbian learning, bioplausibility, dynamical systems theory, etc?) What is a sensible cutoff?
	\item Does it make sense (it is valid?) to base initial weight distribution on previous runs?
  \end{itemize}

\end{frame}
\footnotesize
\begin{frame}{A possible thesis structure}
  \begin{itemize}[label=--]

    \item Introduction (Explanation of concepts, previous work, relevancy of this work)
    \begin{itemize}[label=--]
    \item No formal notation yet, only intuitions
    \item What can I assume is known by reader? MSc. graduate level? Or do I need to re-explain ANNs?
  	\end{itemize}
  	\item Method (data, Bellec's e-prop, my own improvements)
  	\begin{itemize}[label=--]
    \item Adhere to Bellec's formal notation.
    \item Exact reproducibility is paramount.
    \item Can I intersperse the method with theoretical background?
    \item Separate subsection under methods for optional settings (e.g. firing rate regularization). Bellec manages this by putting them in ``supplementary notes''.
    \item Hyperparameter sweep procedure.
  	\end{itemize}
  	\item Results
  	\item Discussion (interpreting results in light of other research, ideas for future work)
  	\item Conclusion (summary, logical connection to Introduction)
  	\item Appendix (supplementary figures, everything that may be interesting or illustrate findings described in main text)

  \end{itemize}

\end{frame}
\end{document}
